\hspace{1 em} In one sentence - future for biometric authentication looks promising. 
With continuous advancements in technology, biometric authentication is expected to become more secure, reliable, and user-friendly.
\subsection{Enhanced security}
\hspace{1 em} Multi-modal biometrics looks to be the way of future authentication. Combining multiple biometric modalities (such as fingerprint, facial recognition, iris scanning, voice recognition, etc.) for enhanced accuracy and security will become more common. Multi-modal 
systems can provide stronger authentication compared to single-modal systems. 
Continuous authentication is another area of development. Instead of a one-time
authentication process, continuous authentication systems monitor user behavior
and biometric data throughout the session to ensure that the user remains
authenticated. This can help prevent unauthorized access in case of a security
breach or a change in user behavior.
\subsection{Improved user experience}
\hspace{1 em} Biometric authentication is expected to become more user-friendly and
convenient. Advancements in biometric sensors and algorithms will make the
authentication process faster and more accurate. For example, the use of 3D facial
recognition technology can improve the accuracy of facial recognition systems and
reduce the risk of spoofing. Additionally, biometric authentication systems are
expected to become more integrated with other technologies, such as smart home
devices, wearables, and IoT devices, to provide a seamless user experience.
\subsection{Widespread adoption}
\hspace{1 em} Biometric authentication is likely to see increased adoption across various
industries and applications. The rise of digital banking and e-commerce is expected
to drive the demand for biometric authentication solutions to enhance security and
combat fraud. Additionally, the integration of biometric authentication into
smartphones, laptops, and other consumer devices will make it more accessible to
the general public. As biometric technology becomes more affordable and
user-friendly, it is expected to become a standard feature in many digital products
and services.
