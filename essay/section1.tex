\hspace{1 em} Biometric authentication in means of cybersecurity is a process that
verifies users identity by evaluating their unique physical or behavioral features. Biometric
authentication typically is more secure and reliable than traditional methods of authentication. 

General trend for biometric authentication market share growth is exponential. According
to Markets and Markets report\cite{marketsandmakets} biometric system market is on trend to 
double by year 2027. According to the report, main driver for this growth is increasing use of biometric technology in consumer electronics for authentication and identification purposes.
Also, huge driver for biometric authentication popularity was COVID-19 pandemic, 
which increased demand for contactless authentication methods. Moreover, biometric authentication
is getting integrations with various machine learning (ML) and artificial intelligence (AI)
models\cite{BRI}. As a result, systems are becoming more accurate and reliable.

\subsection{Types of biometric systems}

As an alterantive to standard password authentication, these are the most popular biometric based alterantives.

Fingerprint Recognition:
Arguably the most familiar form of biometric authentication, fingerprint recognition
 relies on capturing and analyzing the unique patterns present in an individual's 
 fingerprints. Widely used in smartphones, laptops, and access control systems,
 fingerprint recognition offers a balance of security and user convenience. 
 However, concerns about spoofing and privacy have led to advancements in this 
 technology, such as the incorporation of liveness detection.

Facial Recognition:
The advent of facial recognition technology has sparked both excitement and 
controversy. By analyzing facial features such as the distance between the eyes, 
nose, and mouth, facial recognition systems can accurately identify individuals. 
From unlocking smartphones to enhancing surveillance systems, this technology 
offers unparalleled convenience. Nevertheless, ethical debates surrounding privacy 
invasion and potential biases in algorithmic decision-making underscore the need 
for responsible deployment and regulation.

Iris Recognition:
Delving deeper into the realm of biometrics, iris recognition stands out for its 
accuracy and reliability. By capturing the intricate patterns of the iris, which 
are unique to each individual, this technology offers robust authentication. 
Commonly used in high-security environments such as border control and government 
facilities, iris recognition boasts low false acceptance rates. However, concerns 
about user acceptance, cost, and environmental factors such as lighting conditions
 pose challenges to widespread adoption.

Voice Recognition:
The human voice, with its distinct pitch, tone, and cadence, serves as a 
powerful biometric identifier. Voice recognition technology analyzes these vocal 
characteristics to verify the identity of individuals. From call centers to smart
 home devices, voice authentication offers a seamless user experience. Nevertheless, 
 factors such as background noise, variations in speech patterns, and the potential 
 for voice recordings raise concerns about security and reliability.

Behavioral Biometrics:
Beyond physical traits, behavioral biometrics focus on unique patterns in human 
behavior. This includes typing rhythm, mouse movements, and even gait analysis. 
By analyzing these behavioral cues, authentication systems can passively verify 
users without requiring explicit actions. While behavioral biometrics offer 
continuous authentication and resistance to spoofing, concerns about user privacy 
and the need for transparent consent mechanisms loom large.

There are even more specialized biometric metrics that can be used for authentication
, such as retinal scans, hand geometry analysis, scent identification, finger vein scanning,
thermodynamic biometric matching, gait identification, keystroke matching, ear shape analysis,
signature confirmation.

\subsection{But how does biometric authentication work?}

\hspace{1 em} The process of authentication independantly on type of biometric system is quite 
similar. The first stop on authentication roadmap is to map out the data of the selected
biometric trait. Eligible biometric traits involve several different parts of the human
body, such as fingerprints, iris, voice, gait, facial features, DNA and many more features.

Enrollment process is the first step in biometric authentication. Before authentication method
can be employed, the user firstly must enroll their biometric data into a secure system.
During this process, a user's biometric traits are captured and converted into digital templates using specialized hardware such as fingerprint scanners, iris scanners, or cameras. For instance, in fingerprint authentication, the unique ridges and 
patterns of a person's fingerprint are scanned and converted into a mathematical 
algorithm that represents the fingerprint's unique characteristics. This algorithm 
is then securely stored in a database for future comparison.

Authentication process is a second part of biometric authentication. When a user 
attempts to access a system or device that utilizes biometric authentication, the 
system prompts them to provide their biometric data for verification. The captured 
biometric data is then compared with the stored template in the database. In 
the case of fingerprint authentication, for example, the user's presented 
fingerprint is scanned, and its unique features are compared with the stored 
algorithm. If the presented data closely matches the stored template within an 
acceptable margin of error, access is granted. Otherwise, the authentication fails.

Biometric authentication systems employ robust encryption techniques to ensure the security and privacy of biometric data. Instead of storing raw biometric data, systems typically store encrypted templates that cannot be reverse-engineered to reconstruct the original biometric information. Additionally, stringent measures are in place to protect against unauthorized access to the biometric database.

Moreover, many biometric authentication systems incorporate liveness detection mechanisms to prevent spoofing attempts. These mechanisms assess the liveliness of the presented biometric data, such as detecting pulse in fingerprint scans or monitoring facial movements during facial recognition.

