\hspace{1 em} Looking at biometrics from completely technical perspective it could be defined as the measurament and evaluation of the 
individual specimens unique biological parameters. By looking at the definition, this technology does not seem as it  would be 
available or even created for humans. However, this is not the case, as biometrics is used in various fields, 
such as authentication, identification, and access control. Biometrics encompasses the science and technology of measuring and analyzing 
these distinct traits, enabling accurate and reliable authentication processes.  Many people do not even realize how much of their
daily activities include biometrics.

\subsection{Biometrics for authentification}
\hspace{1 em} Biometric authentication is a process of verifying the identity of an individual 
based on their unique biological characteristics. Biometric authentication relies on a diverse array of
 physiological and behavioral traits, each
 possessing unique characteristics that distinguish individuals from one another.  Physiological biometrics 
 include fingerprints, iris patterns, facial features, and DNA, while behavioral biometrics encompass
  patterns such as typing rhythm, voice patterns, and gait. These traits are captured and processed using 
specialized sensors and algorithms, converting them into digital 
 templates for comparison and verification (Jain, Ross, \& Nandakumar, 2016).

 The adoption of biometric authentication spans various domains, offering enhanced security and convenience. 
 In access control systems, biometrics replace traditional methods like passwords or access cards, mitigating 
 ecurity risks and improving user experience (Li, Ye, \& Wang, 2019). Biometrics also find applications in 
 financial services, border security, healthcare, and consumer electronics, where they play pivotal roles in 
 fraud prevention, identity verification, and user authentication (Amin, Malik, \& Gohar, 2020).

 \subsection{Advantages of biometric authentication}
    \hspace{1 em} Biometric authentication offers several advantages over traditional authentication methods.
    First of all, it is more secure due to reliance on unique biological traits or behaviours, which are difficult
    to replicate or forge. Moreover, biometric authentication tends to be more accurate and reliable than other
    methods of authentication, once the trait is set up, it is there to stay, but typing in the password 10 times 
    in a row with no mistakes could be challenging.

    Biometric methods of authentication offers increased convenience. Users do not need to
    memorize complex passwords, carry access keys or cards. Including biometrics in the authentication process,
    could also improve the ergonomics of the devices, for example fingerprint scanner on the back of the phone:
    unlocking the device while pulling it out of the pocket. Biometric authentication systems also reduce the
    administrative burden associated with managing passwords, access cards, and other authentication tokens.

    Biometric authentication raises users awareness while enlisting for new services - the process of being
    verified by fulfilling the tasks while on camera requires users to be present and aware of the process.

    Identity theft is becoming a growing concern in the digital age, with cybercriminals exploiting vulnerabilities
    in traditional authentication methods to gain unauthorized access to sensitive information. Biometric authentication
    offers a robust defense against identity theft, as biometric traits are unique to each individual and cannot be easily
    replicated or stolen. However, the risk of identity theft emerges again, with the development of machine learning
    algorithms, that are getting better at replicating biometric traits, such as facial features.